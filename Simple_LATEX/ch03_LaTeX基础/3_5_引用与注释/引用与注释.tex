\documentclass[]{ctexart}
\usepackage{lastpage}
\usepackage{amsmath}
\begin{document}
	出师表\footnote{title}
	
	\label{section:author}诸葛亮〔两汉〕
	
	先帝创业未半而中道崩殂,今天下三分,益州疲弊,此诚危急存亡之秋也。然侍卫之臣不懈于内,忠志之士忘身于外者,盖追先帝之殊遇,欲报之于陛下也。诚宜开张圣听,以光先帝遗德,恢弘志士之气,不宜妄自菲薄,引喻失义,以塞忠谏之路也。
	
	宫中府中,俱为一体;陟罚臧否,不宜异同。若有作奸犯科及为忠善者,宜付有司论其刑赏,以昭陛下平明之理,不宜偏私,使内外异法也。
	
	侍中、侍郎郭攸之、费祎、董允等,此皆良实,志虑忠纯,是以先帝简拔以遗陛下。愚以为宫中之事,事无大小,悉以咨之,然后施行,必能裨补阙漏,有所广益。
	
	将军向宠,性行淑均,晓畅军事,试用于昔日,先帝称之曰能,是以众议举宠为督。愚以为营中之事,悉以咨之,必能使行阵和睦,优劣得所。
	
	亲贤臣,远小人,此先汉所以兴隆也;亲小人,远贤臣,此后汉所以倾颓也。先帝在时,每与臣论此事,未尝不叹息痛恨于桓、灵也。侍中、尚书、长史、参军,此悉贞良死节之臣,愿陛下亲之信之,则汉室之隆,可计日而待也。
	
	臣本布衣,躬耕于南阳,苟全性命于乱世,不求闻达于诸侯。先帝不以臣卑鄙,猥自枉屈,三顾臣于草庐之中,咨臣以当世之事,由是感激,遂许先帝以驱驰。后值倾覆,受任于败军之际,奉命于危难之间,尔来二十有一年矣。
	
	先帝知臣谨慎,故临崩寄臣以大事也。受命以来,夙夜忧叹,恐托付不效,以伤先帝之明;故五月渡泸,深入不毛。今南方已定,兵甲已足,当奖率三军,北定中原,庶竭驽钝,攘除奸凶,兴复汉室,还于旧都。此臣所以报先帝而忠陛下之职分也。至于斟酌损益,进尽忠言,则攸之、祎、允之任也。
	
	愿陛下托臣以讨贼兴复之效,不效,则治臣之罪,以告先帝之灵。若无兴德之言,则责攸之、祎、允等之慢,以彰其咎;陛下亦宜自谋,以咨诹善道,察纳雅言,深追先帝遗诏。臣不胜受恩感激。今当远离,临表涕零,不知所言。
	
	
	出师表原文及翻译
	
	出师表介绍:
	
	《出师表》是三国时期蜀汉丞相诸葛亮在北伐中原之前给后主刘禅上书的表文,阐述了北伐的必要性以及对后主刘禅治国寄予的期望,言辞恳切,写出了诸葛亮的一片忠诚之心。历史上有《前出师表》。至于三国演义中的后出师表,并没有证实。通常所说的《出师表》一般指《前出师表》。表,古代向帝王上书陈情言事的一种文体。
	
	出师表原文:
	
	先帝创业未半而中道崩殂(cú),今天下三分,益州疲(pí)弊,此诚危急存亡之秋也。然侍卫之臣不懈(xiè)于内,忠志之士忘身于外者,盖追先帝之殊遇,欲报之于陛下也。诚宜开张圣听,以光先帝遗(yí)德,恢弘志士之气,不宜妄自菲薄,引喻失义,以塞(sè)忠谏之路也。
	
	宫中府中,俱为一体,陟(zhì )罚臧(zāng)否(pǐ),不宜异同。若有作奸犯科及为忠善者,宜付有司论其刑赏,以昭陛下平明之理,不宜偏私,使内外异法也。
	
	侍中、侍郎郭攸(yōu)之、费祎(yī)、董允等,此皆良实,志虑忠纯,是以先帝简拔以遗(wèi)陛下。愚以为宫中之事,事无大小,悉以咨之,然后施行,必能裨(bì)补阙漏,有所广益。
	
	将军向宠,性行(xíng)淑均,晓畅军事,试用于昔日,先帝称之曰能,是以众议举宠为督。愚以为营中之事,悉以咨之,必能使行(háng )阵和睦,优劣得所。
	
	亲贤臣,远小人,此先汉所以兴隆也;亲小人,远贤臣,此后汉所以倾颓也。先帝在时,每与臣论此事,未尝不叹息痛恨于桓(huán)、灵也。侍中、尚书、长(zhǎng)史、参军,此悉贞良死节之臣,愿陛下亲之信之,则汉室之隆,可计日而待也。
	
	臣本布衣,躬耕于南阳,苟全性命于乱世,不求闻(wén)达于诸侯。先帝不以臣卑(bēi)鄙(bǐ),猥(wěi)自枉屈,三顾臣于草庐之中,咨臣以当世之事,由是感激,遂许先帝以驱驰。后值倾覆,受任于败军之际,奉命于危难之间,尔来二十有(yòu)一年矣! 【“有”是通假字,通“又”,跟在数词后面表示约数。所以读yòu】
	
	先帝知臣谨慎,故临崩寄臣以大事也。受命以来,夙(sù)夜忧叹,恐托付不效,以伤先帝之明,故五月渡(dù)泸,深入不毛。今南方已定,兵甲已足,当奖率三军,北定中原,庶(shù)竭驽(nú)钝,攘(rǎng)除奸凶,兴复汉室,还于旧都。此臣所以报先帝而忠陛下之职分也。至于斟酌损益,进尽忠言,则攸之、祎、允之任也。
	
	愿陛下托臣以讨贼兴复之效,不效,则治臣之罪,以告先帝之灵。若无兴德之言,则责攸之、祎、允等之慢,以彰其咎(jiù)。陛下亦宜自谋,以咨诹(zōu)善道,察纳雅言,深追先帝遗诏。臣不胜受恩感激!
	
	今当远离,临表涕零,不知所云。
	
	\ref{section:author} 	% no output
	
	Author is in page: \pageref{section:author} of total \pageref{LastPage} \\
	
	\footnote{This is one footnote.}
	
	出师表翻译:
	
	先帝开创的大业未完成一半却中途去世了。现在天下分为三国,益州地区民力匮乏,这确实是国家危急存亡的时期啊。不过宫廷里侍从护卫的官员不懈怠,战场上忠诚有志的将士们奋不顾身,大概是他们追念先帝对他们的特别的知遇之恩(作战的原因),想要报答在陛下您身上。(陛下)你实在应该扩大圣明的听闻,来发扬光大先帝遗留下来的美德,振奋有远大志向的人的志气,不应当随便看轻自己,说不恰当的话,以致于堵塞人们忠心地进行规劝的言路。
	
	\begin{quote}皇宫中和朝廷里的大臣,本都是一个整体,奖惩功过,好坏,不应该有所不同。如果有做奸邪事,犯科条法令和忠心做善事的人,应当交给主管的官,判定他们受罚或者受赏,来显示陛下公正严明的治理,而不应当有偏袒和私心,使宫内和朝廷奖罚方法不同。\end{quote}
	
	侍中、侍郎郭攸之、费祎、董允等人,这些都是善良诚实的人,他们的志向和心思忠诚无二,因此先帝把他们选拔出来辅佐陛下。我认为(所有的)宫中的事情,无论事情大小,都拿来跟他们商量,这样以后再去实施,一定能够弥补缺点和疏漏之处,可以获得很多的好处。
	
	\begin{quotation}将军向宠,性格和品行善良公正,精通军事,从前任用时,先帝称赞说他有才干,因此大家评议举荐他做中部督。我认为军队中的事情,都拿来跟他商讨,就一定能使军队团结一心,好的差的各自找到他们的位置。\end{quotation}
	
	亲近贤臣,疏远小人,这是西汉之所以兴隆的原因;亲近小人,疏远贤臣,这是东汉之所以衰败的原因。先帝在世的时候,每逢跟我谈论这些事情,没有一次不对桓、灵二帝的做法感到叹息痛心遗憾的。侍中、尚书、长史、参军,这些人都是忠贞诚实、能够以死报国的忠臣,希望陛下亲近他们,信任他们,那么汉朝的兴隆就指日可待了。
	
	\begin{verse}我本来是平民,在南阳务农亲耕,在乱世中苟且保全性命,不奢求在诸侯之中出名。先帝不因为我身份卑微,见识短浅,降低身份委屈自己,三次去我的茅庐拜访我,征询我对时局大事的意见,我因此十分感动,就答应为先帝奔走效劳。后来遇到兵败,在兵败的时候接受任务,在危机患难之间奉行使命,那时以来已经有二十一年了。\end{verse}
	
	先帝知道我做事小心谨慎,所以临终时把国家大事托付给我。接受遗命以来,我早晚忧愁叹息,只怕先帝托付给我的大任不能实现,以致损伤先帝的知人之明,所以我五月渡过泸水,深入到人烟稀少的地方。现在南方已经平定,兵员装备已经充足,应当激励、率领全军将士向北方进军,平定中原,希望用尽我平庸的才能,铲除奸邪凶恶的敌人,恢复汉朝的基业,回到旧日的国都。这就是我用来报答先帝,并且尽忠陛下的职责本分。至于处理事务,斟酌情理,有所兴革,毫无保留地进献忠诚的建议,那就是郭攸之、费祎、董允等人的责任了。
	
	希望陛下能够把讨伐曹魏,兴复汉室的任务托付给我,如果没有成功,就惩治我的罪过,(从而)用来告慰先帝的在天之灵。如果没有振兴圣德的建议,就责罚郭攸之、费祎、董允等人的怠慢,来揭示他们的过失;陛下也应自行谋划,征求、询问治国的好道理,采纳正确的言论,以追念先帝临终留下的教诲。我感激不尽。
	
	今天(我)将要告别陛下远行了,面对这份奏表禁不住热泪纵横,也不知说了些什么。\marginpar{不错的文章}\\
	
	\begin{minipage}{\linewidth}
		\begin{tabular}{1}
			Footnote Sample \footnotemark.
		\end{tabular}
		\footnotetext{no need more}
	\end{minipage}
	
\end{document}

