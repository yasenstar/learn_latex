\documentclass[]{report}
	\title{LaTeX\thanks{nice topic}}
	\author{Yasen from wikepedia\thanks{thank you}}
	\date{November 1, 2023\thanks{key date}}
\begin{document}
	\maketitle
	
	\tableofcontents
	
	\chapter{starting}LaTeX (/ˈlɑːtɛk/ LAH-tek or /ˈleɪtɛk/ LAY-tek,
	[2][Note 1] often
	stylized as LATEX) is a software system for document
	preparation.[3] When writing, the writer uses plain text as
	opposed to the formatted text found in WYSIWYG word
	processors like Microsoft Word, LibreOffice Writer and Apple
	Pages. The writer uses markup tagging conventions to define the
	general structure of a document, to stylise text throughout a
	document (such as bold and italics), and to add citations and
	cross-references. A TeX distribution such as TeX Live or
	MiKTeX is used to produce an output file (such as PDF or DVI)
	suitable for printing or digital distribution.
	LaTeX is widely used in academia[4][5]
	for the communication
	and publication of scientific documents in many fields,
	including mathematics, computer science, engineering, physics,
	chemistry, economics, linguistics, quantitative psychology,
	philosophy, and political science. It also has a prominent role in
	the preparation and publication of books and articles that
	contain complex multilingual materials, such as Arabic and
	Greek.
	
	\chapter{Start actual content} LaTeX uses the TeX typesetting program for
	formatting its output, and is itself written in the TeX macro
	language.
	LaTeX can be used as a standalone document preparation
	system, or as an intermediate format. In the latter role, for
	example, it is sometimes used as part of a pipeline for
	translating DocBook and other XML-based formats to PDF. The
	typesetting system offers programmable desktop publishing features and extensive facilities for
	automating most aspects of typesetting and desktop publishing, including numbering and crossreferencing of tables and figures, chapter and section headings, graphics, page layout, indexing and
	bibliographies.
	
	\section{Starting Latex}Like TeX, LaTeX started as a writing tool for mathematicians and computer scientists, but even from
	early in its development, it has also been taken up by scholars who needed to write documents that
	include complex math expressions or non-Latin scripts,[7]
	such as Arabic, Devanagari and Chinese.
	[8]
	LaTeX is intended to provide a high-level, descriptive markup language that accesses the power of
	TeX in an easier way for writers. In essence, TeX handles the layout side, while LaTeX handles the
	content side for document processing. LaTeX comprises a collection of TeX macros and a program to
	process LaTeX documents, and because the plain TeX formatting commands are elementary, it
	provides authors with ready-made commands for formatting and layout requirements such as chapter
	
	\subsection{small part Latex}
	headings, footnotes, cross-references and bibliographies.
	11/7/23, 12:49 PM LaTeX - Wikipedia
	https://en.wikipedia.org/wiki/LaTeX 2/9
	LaTeX was originally written in the early 1980s by Leslie Lamport at SRI International. \paragraph{new one}
	[9] The current
	version is LaTeX2e (stylised as LATEX 2ε
	), first released in 1994 but incrementally updated starting in
	2015. This update policy replaced earlier plans for a separate release of LaTeX3 (LATEX3), which had
	been in development since 1989.[10] LaTeX is free software and is distributed under the LaTeX Project
	Public License (LPPL).
	[11]
	LaTeX was created in the early 1980s by Leslie Lamport when he was working at SRI. He needed to
	write TeX macros for his own use and thought that with a little extra effort, he could make a general
	package usable by others. Peter Gordon, an editor at Addison-Wesley, convinced him to write a LaTeX
	user's manual for publication (Lamport was initially skeptical that anyone would pay money for it);[12]
	it came out in 1986
	[3] and sold hundreds of thousands of copies.[12] Meanwhile, Lamport released
	versions of his LaTeX macros in 1984 and 1985. On 21 August 1989, at a TeX Users Group (TUG)
	meeting at Stanford, Lamport agreed to turn over maintenance and development of LaTeX to Frank
	Mittelbach. 
	
	\section{Using Latex}Frank Mittelbach, along with Chris Rowley and Rainer Schöpf, formed the LaTeX3 team;
	in 1994, they released LaTeX2e, the current standard version. LaTeX3 has since been cancelled with
	features intended for that version being back-ported to LaTeX2e since 2018.[10]
	LaTeX attempts to follow the design philosophy of separating presentation from content, so that
	authors can focus on the content of what they are writing without attending simultaneously to its
	visual appearance. In preparing a LaTeX document, the author specifies the logical structure using
	simple, 
	
	\paragraph{new par}familiar concepts such as chapter, section, table, figure, etc., and lets the LaTeX system
	handle the formatting and layout of these structures. As a result, it encourages the separation of the
	layout from the content — while still allowing manual typesetting adjustments whenever needed. This
	concept is similar to the mechanism by which many word processors allow styles to be defined
	globally for an entire document, or the use of Cascading Style Sheets in styling HTML documents.
	The LaTeX system is a markup language that handles typesetting and rendering,[13] and can be
	arbitrarily extended by using the underlying macro language to develop custom macros such as new
	environments and commands. Such macros are often collected into packages, which could then be
	made available to address some specific typesetting needs such as the formatting of complex
	mathematical expressions or graphics (e.g., the use of the align environment provided by the
	amsmath package to produce aligned equations).
	In order to create a document in LaTeX, a user first creates a file, such as document.tex, typically
	using a text editor. The user then gives their document.tex file as input to the TeX program (with the
	LaTeX macros loaded), which prompts TeX to write out a file suitable for onscreen viewing or
	printing.[14] This write-format-preview cycle is one of the chief ways in which working with LaTeX
	differs from the What-You-See-Is-What-You-Get (WYSIWYG) style of document editing. It is similar
	to the code-compile-execute cycle known to computer programmers. Today, many LaTeX-aware
	editing programs make this cycle a simple matter through the pressing of a single key, while showing
	History
	Typesetting system
	11/7/23, 12:49 PM LaTeX - Wikipedia
	https://en.wikipedia.org/wiki/LaTeX 3/9
	the output preview on the screen beside the input window. Some online LaTeX editors even
	automatically refresh the preview,[15][16][17] while other online tools
\end{document}
