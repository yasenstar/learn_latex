\documentclass[]{article}
\usepackage{graphicx}
\usepackage{wrapfig}
\begin{document}
	\begin{wrapfigure}{i}{0.8\linewidth}
		
		It may be noted that the width of the image included was specified relative to width of the text (\textwidth). It is a good idea to use relative sizes to define lengths (height, width, etc), particularly when using wrapfigure.
		
		\begin{center}
			\includegraphics[width=0.4\linewidth]{house.png}
		\end{center}
		\caption{HOUSE} 
		
		In the example above, the figure covers exactly half of the the textwidth, and the actual image uses a slightly smaller width, so that there is a pleasing small white frame between the image and the text. The image should always be smaller (less wide) than the wrap, or it will overrun the text.
	\end{wrapfigure}
\end{document}